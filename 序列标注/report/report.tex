\documentclass{article} % For LaTeX2e
\usepackage{nips13submit_e,times}
\usepackage{hyperref}
\usepackage{url}
\usepackage{xeCJK}
\setCJKmainfont{宋体}
\renewcommand{\baselinestretch}{1.3}
%\documentstyle[nips13submit_09,times,art10]{article} % For LaTeX 2.09


\title{LSTM-CRF模型在序列标注问题上的应用\\
\begin{normalsize} FudanNLP-nlp-beginner 任务四探究报告 \end{normalsize}}
\subtitle{}

\author{
戴宁
}

\newcommand{\fix}{\marginpar{FIX}}
\newcommand{\new}{\marginpar{NEW}}

\hypersetup{
colorlinks=true,
linkcolor=black
}

\nipsfinalcopy % Uncomment for camera-ready version

\begin{document}
\maketitle
%\tableofcontents
%\newpage

\section{Introduction}
\qquad 序列标注问题,即给定观测序列$x = (x_1, x_2, ..., x_n)$,需要预
测相应的标注序列$y = (y_1, y_2, ..., y_n)$,其中$n$是序列长度。具体的,
学习系统需要建模$p(y|x)$,在给定x的情况下,学习系统会将$y^* = argmax_y
\; p(y|x)$作为对当前观测序列所对应标注序列的预测。 

\subsection{LSTM-CRF}
\qquad 条件随机场(CRF)用于建模条件概率$p(y|x)$,在序列标注问题中通常使
用线性链条件随机场,其形式如下:$$p(y|x;w) = \frac{exp(w \cdot F(x,
  y))}{\sum_{y^{'}} exp(w \cdot F(x,y^{'}))}$$

\qquad 其中$w$为需要学习的参数向量,而$F(x,y) =
\sum_{i=1}^{n}f(x, i, y_{i-1}, y_i)$是局部特征向量之和。

\qquad 传统的机器学习就需要对特征函数$f(x, i, y_{i-}, y_i)$进行精心的
设计从而达到较好的效果。

\qquad 对于深度学习来说,由于其拥有较强的学习特征表示的能力,故常考虑
用神经网络代替人工设计的特征。在使用神经网络时,考虑如下方式定义的CRF:
$$p(y|x) = \frac{exp(\sum_{i=1}^{n}A_{y_{i-1}, y_i} + P_{i, y_i}(x))}
{\sum_{y^{'}}exp(\sum_{i=1}^{n}A_{y^{'}_{i-1}, y^{'}_i} + P_{i,
    y^{'}_i}(x))}$$

\qquad 其中$A$为转移矩阵,$A_{y_{i-1}, y_i}$用于评价从标记$y_{i-1}$转移到标
记$y_i$的优劣,而$P_{i, y_i}(x)$表示观测序列为$x$的前提下,在第$i$个位置
使用标记$y_i$的优劣。其中$A$为通过学习得到的参数矩阵,$P(x)$为神经网络
需要学习到的映射。

\qquad 虽然$P(x)$可以为任意类型的神经网络,但考虑到$x$是序列型输入,故
主要构架使用RNN(LSTM、GRU)进行建模。

\section{Model}
\qquad 模型参照[3]中提出的char-level-feature + LSTM-CRF模型,本文中
char-level-feature的提取使用的是GRU网络,同时使用IOBES格式的标记。

\subsection{embedding}
\subsubsection{word level}
\qquad 通过[2]中的对比,发现GloVe算法得到的单词的词嵌入更加适合命名实
体识别任务,故使用与[2]中相同的由GloVe预训练得到100维词嵌入向量。同时
通过[2]中的单词分布分析得到,在开发集和测试集中有大量训练集中未出现的
单词,故使用所有400000个预训练过的单词作为词表,同时忽略大小写。

\qquad 对于未出现在词表中的词,用标识符<UNK>代替。<UNK>的embedding由词
表中所有词的embedding取平均得到。

\qquad 以上embedding在训练过程中可以被梯度下降调整。

\subsubsection{char level}
\qquad 由于英语的语言特点,很多词型上的特征会对NER任务提供很多重要的信
息,故考虑使用一个额外的GRU网络来提取每个单词词型的特征。

\qquad 字符级的词表为训练集中所有出现过的字符,区分大小写。这样一来在
单词级忽略掉的大小写信息可以由字符级的特征刻画。每个字符的词嵌入向量在
$[-1.0, 1.0]$直接由均匀分布生成,并且设置为可调整,通过训练过程中进一
步学习得到。

\qquad 字符级词表不设<UNK>标识,未出现在词表中的字符直接忽略不计。

\qquad 每个单词的字符通过一个隐状态数为25的双向GRU网络,其前向与后向最终状态一起作为当前
单词embedding的一部分。

\qquad 通过拼接word level和char level的特征,最终每个单词由150维特征表
示。

\subsection{LSTM-CRF}
\qquad 每个句子由前面的步骤得到单词的向量表示,作为双向LSTM单元的输入。
然后将前向与后向得到的输出拼接后通过一个全连接层,投影成标记对应词表的大小,作为对每
个位置使用每个标记的打分。

\qquad 之后通过CRF层与转移矩阵一起对当前的标记序列打分。


\section{Training Detail}

\subsection{hyperparameters}

\qquad 字符词嵌入向量为30维,单词词嵌入向量为100维

\qquad GRU的隐单元数为25,LSTM的隐单元数为256。

\qquad mini batch的大小为128,使用Adam算法计算梯度,其中$\beta_1=0.9,
\beta_2=0.999$。

\qquad 最终选择在开发集上F1分数最高的参数作为模型最终参数。

\subsection{dropout}
\qquad 在字符特征输入到GRU之前、单词特征输入到LSTM之前、LSTM输出进入全
连接层之前都进行了dropout操作,keep\_prob为0.4或0.5。

\subsection{gradient clipping}
\qquad 为了防止RNN在训练过程中出现梯度爆炸的问题,在训练过程中对模
型进行了梯度截断,公式如下:
$$dW_{cliped} = \frac{dW_{raw} * max\_norm}{max(||dW_{raw}||_2, \;max\_norm)}$$

\qquad 其中$dW_{raw}$代表所有变量的原始梯度,$dW_{cliped}$代表所有变量的被截断
后梯度,$||dW_{raw}||_2$代表$dW_{raw}$的L2范数,max\_norm为截断的阈值
(设置为5)。

\subsection{learning rate decay}
\qquad 为了防止训练后期模型震荡,在模型训练整个过程的前$1/2$学习率为初始设定的学习率,而训练过程的后$1/2$,采用线性衰减的策略,从初始学习率0.002一直衰减到0为止。

\section{Result}
\qquad 本文分别实验了不加char-level-feature和附加char-level-feature的
LSTM-CRF模型

\subsection{without char-level-feature}
\qquad 一开始没有注意到[2]中对单词分布的分析,直接将训练集出现过的单词
作为词表,模型的泛化能力很差,在训练集上能够到达接近$100\%$的F1分数,
但是在开发集上就只有$85\%$左右的F1,在测试集上更是$80\%$都不到。

\qquad 将所有400000个单词作为词表之后,在测试集上的F1分数能够达到
$85.54\%$(使用IOB1格式标记)。

\qquad 将标记改成IOBES格式后,模型的F1分数能够达到$87.25\%$

\subsection{with char-level-feature}
\qquad 在前面的基础上,加上了char-level-feature,模型在测试集上的F1分数直接升到
了$90.51\%$

\qquad 尝试对LSTM部分进行叠加,在2层LSTM的情况下达到了$90.70\%$的
F1分数。

\qquad 最后考虑到,由于训练集单词数量较小,在训练过程中对GloVe向量的再调整
可能造成模型的过拟合,故尝试将单词的词嵌入部分设置为不可被调整。

\qquad 实验后发现对最终结果没有太大影响,在相同参数下F1分数反而稍微下
降了些。


\subsubsection*{References}
\small{
[1] Zhiheng Huang. ; Wei Xu. ; \& Kai Yu. (2015) {\it Bidirectional LSTM-CRF 
Models for Sequence Tagging.} Retrieved from
https://arxiv.org/abs/1508.01991

[2] Xuezhe Ma. ; \& Eduard Hovy. (2016) {\it End-to-end Sequence Labeling 
via Bi-directional LSTM-CNNs-CRF.} Retrieved from 
https://arxiv.org/abs/1603.01354

[3] Guillaume Lample. ; Miguel Ballesteros. ; Sandeep Subramanian. ;
Kazuya Kawakami. ; \& Chris Dyer. (2016) {\it Neural Architectures for Named
  Entity Recognition.} Retrieved from 
https://arxiv.org/abs/1603.01360

\end{document}
